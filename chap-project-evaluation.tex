\chapter{Project Evaluation}
\label{chap:project-evaluation}

\subsection{COVID-19}
Due to COVID-19 we could not carry out in person supervisor meetings, which could have had an impact on the feedback between the group and the supervisor. The group does however think that the online meetings were utilised to its full capabilities, and communication was still possible, even outside of the meetings. COVID also had an impact on the ability to have people test our software in person, which lead to us having worse feedback than was optimal. As our application has great focus on the user, it would have been ideal if a user test session was possible. Lastly, being stuck at home also had an effect on the progress of the project, as scheduling and maintaining meetings between the group proved difficult.
\subsection{Development}
During this project the group used an iterative development model. Overall the development process was effective but unstructured, and a more structured development method could have been useful. This especially comes to mind when considering the size of the project, meaning that it was difficult to manage everything with the lack of tools used throughout. Potential tools and methods that could have been used to help structure is the kanban board, taking elements from Scrum such as the daily meetings, or a more structured development method like AUP or RUP.\\
Throughout the project there were multiple times where the group was distracted and busy with other projects and courses due this semester. This meant that it was not possible to devote our full time to this project and this lead to some weeks were little to none progress were made.\\
Lastly, when working with new subjects such as SPARQL and RDF, there is big reason to do lots of research in the first part of the project. This of course helped us develop a better system, but there were however feelings of being stuck.
\subsection{Product}
Overall the group is satisfied with the product outcome of the project. The product hit all the major requirements we wanted and the few requirements we did not manage were due to unforeseen issues we could not solve in time. The implementation of the product helped us learn a lot about JavaScript development, the RDF and SPARQL connection and how to work in this type of project. The group was not used to work on open ended projects and were not used to work on projects of this size in pairs. For future projects we now also know what is required to work with an unknown subject, and will not be discouraged by the feeling of going nowhere, as we know that progress is slow.