\chapter{Method and Tools}
\label{chap:method-and-tools}

\section{Iterative development}
The iterative process is the development in different iterations each with their own well defined goal for what the purpose of set iteration is. This is referred to as sprints in some development processes. {\color{red}missing\\}

\subsection{Backlog}
The backlog has a list of all requirements that the software needs to be completed. This is essential for scoping iterations, tracking progress, and making sure that our software solution includes all the necessary tasks.

\section{Pair programming}
While both participants have experience with programming, it can be a good idea to incorporate pair programming. This has especially been helpful for this project as new libraries and concepts have been introduced. Pair programming helps ensure better code practices as the written code naturally comes into discussion. While useful, pair programming was not used at all times during development. As stated, pair programming has its advantages, but can be very time consuming and inefficient.

\section{Github and version control}
Throughout the project, github has been a vital part of development. Github is an online service for handling .git version control without having to write .git code. When developing in teams the need for version control is of considerable importance. This is due to the fact that it allows for each individual to work on two separate branches without any interference.
Furthermore, as this project deals in iterations, version control helps with pushing different working versions of the application. If a critical flaw is detected in new code, a rollback to a working version is always possible.
